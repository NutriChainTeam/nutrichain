\documentclass[11pt,a4paper]{article}
\usepackage[utf-8]{inputenc}
\usepackage[french]{babel}
\usepackage{xcolor}
\usepackage{geometry}
\usepackage{hyperref}
\usepackage{fancyhdr}
\usepackage{longtable}
\usepackage{booktabs}
\usepackage{listings}
\usepackage{xparse}

\geometry{margin=1in}
\pagestyle{fancy}
\fancyhf{}
\fancyhead[L]{NutriChain - Documentation Technique}
\fancyhead[R]{\today}
\fancyfoot[C]{\thepage}

\lstset{
  basicstyle=\ttfamily\small,
  breaklines=true,
  breakatwhitespace=true,
  keywordstyle=\color{blue},
  commentstyle=\color{gray},
  stringstyle=\color{red},
  backgroundcolor=\color{lightgray!20},
  frame=single,
  language=Python,
  morekeywords={render_template, jsonify}
}

\title{\textbf{NutriChain}\\Documentation Technique et Checklist de Démarrage}
\author{Équipe Technique}
\date{\today}

\begin{document}

\maketitle
\tableofcontents
\newpage

% =====================================================
\section{Structure du Projet}
% =====================================================

\subsection{Organisation des fichiers}

\begin{verbatim}
nutrichain_restored/
├── api.py                    # Backend Flask (API + routes HTML)
├── templates/
│   ├── dashboard.html        # Dashboard principal
│   ├── index.html            # Page d'accueil (optionnel)
│   └── ...
├── static/                   # Fichiers CSS, JS, images
├── requirements.txt          # Dépendances Python
├── TECH.md                   # Documentation technique
├── CHECKLIST_START.md        # Checklist de démarrage
└── ...
\end{verbatim}

Flask utilise \texttt{templates/} comme dossier de templates par défaut.

% =====================================================
\section{Routes Flask Disponibles}
% =====================================================

Toutes les routes sont définies dans \texttt{api.py}.

\subsection{GET /}

\begin{itemize}
  \item \textbf{Fonction} : \texttt{index()}
  \item \textbf{Rôle} : Affiche le dashboard principal
  \item \textbf{Code} :
\end{itemize}

\begin{lstlisting}
@app.route("/")
def index():
    return render_template("dashboard.html")
\end{lstlisting}

\textbf{Problème classique} : Si cette route n'existe pas $\Rightarrow$ 404 sur \texttt{http://127.0.0.1:5000/}.

\subsection{GET /nchain\_balance/<account\_id>}

\begin{itemize}
  \item \textbf{Fonction} : \texttt{nchain\_balance(account\_id)}
  \item \textbf{Rôle} : Retourne le solde NCHAIN d'un compte Hedera via Mirror Node (lecture seule)
  \item \textbf{Réponse JSON} :
\end{itemize}

\begin{lstlisting}
{
  "account_id": "0.0.10128148",
  "token_id": "0.0.10136204",
  "symbol": "NCHAIN",
  "balance": 100000000.0
}
\end{lstlisting}

\textbf{En cas d'erreur} (HTTP 502) :

\begin{lstlisting}
{
  "error": "Mirror node error",
  "details": "message d'erreur"
}
\end{lstlisting}

% =====================================================
\section{Intégration Hedera}
% =====================================================

\subsection{Configuration}

Paramètres dans \texttt{api.py} :

\begin{lstlisting}
HEDERA_NETWORK = "testnet"  # ou "mainnet"
NCHAIN_TOKEN_ID = "0.0.10136204"
MIRROR_BASE = "https://mainnet-public.mirrornode.hedera.com"
\end{lstlisting}

\subsection{Client Hedera SDK}

\begin{itemize}
  \item Actuellement \textbf{sans opérateur} (pas de \texttt{setOperator}) pour éviter de stocker la clé privée sur le serveur.
  \item Toutes les infos NCHAIN du dashboard passent par Mirror Node REST.
  \item Pas besoin de clé privée pour les opérations de lecture seule.
\end{itemize}

\subsection{Fonction get\_nchain\_balance}

\begin{lstlisting}
def get_nchain_balance(account_id: str) -> float:
    url = f"{MIRROR_BASE}/api/v1/tokens/0.0.10136204/balances"
    params = {"account.id": account_id}
    r = requests.get(url, params=params, timeout=10)
    r.raise_for_status()
    data = r.json()
    balances = data.get("balances", [])
    if not balances:
        return 0.0
    raw = balances[0].get("balance", 0)
    return raw / 1_000_000.0  # si 6 décimales
\end{lstlisting}

\textbf{Points de vigilance} :
\begin{itemize}
  \item \texttt{NCHAIN\_TOKEN\_ID} doit être passé comme \textbf{string} dans l'URL Mirror Node.
  \item Mauvais token ou mauvais \texttt{MIRROR\_BASE} $\Rightarrow$ 404 ou 5xx.
\end{itemize}

% =====================================================
\section{Dashboard (templates/dashboard.html)}
% =====================================================

\subsection{Grille de statistiques}

\begin{lstlisting}[language=HTML]
<!-- STATS GRID -->
<div class="stats-grid">
  <div class="stat-card">
    <h3 data-i18n="mealsdistributed">Repas distribués</h3>
    <div class="value" id="mealsCount">12 847</div>
    <div class="hint">Basé sur les dons NUTRI</div>
  </div>
  
  <!-- ... autres cartes ... -->
  
  <!-- Carte NCHAIN -->
  <div class="stat-card">
    <h3>Trésorerie NCHAIN</h3>
    <div class="value" id="nchain-balance">Chargement...</div>
    <div class="hint">Solde du compte 0.0.10128148</div>
  </div>
</div>
\end{lstlisting}

\subsection{Script de chargement NCHAIN}

À placer juste avant \texttt{</body>} :

\begin{lstlisting}[language=HTML]
<script>
  async function loadNchainBalance() {
    try {
      const res = await fetch("/nchain_balance/0.0.10128148");
      if (!res.ok) {
        throw new Error("HTTP " + res.status);
      }
      const data = await res.json();
      const el = document.getElementById("nchain-balance");
      if (!el) return;
      el.textContent = data.balance.toLocaleString("fr-FR") 
                       + " " + data.symbol;
    } catch (e) {
      const el = document.getElementById("nchain-balance");
      if (el) {
        el.textContent = "Erreur chargement solde";
      }
      console.error("Error loading NCHAIN balance:", e);
    }
  }

  document.addEventListener("DOMContentLoaded", loadNchainBalance);
</script>
</body>
\end{lstlisting}

Ce script :
\begin{itemize}
  \item Fait un \texttt{GET /nchain\_balance/0.0.10128148}
  \item Lit \texttt{data.balance} et \texttt{data.symbol}
  \item Affiche dans \texttt{<div id="nchain-balance">}
\end{itemize}

% =====================================================
\section{Problèmes Fréquents}
% =====================================================

\begin{longtable}{|p{3cm}|p{3.5cm}|p{4.5cm}|}
\hline
\textbf{Problème} & \textbf{Cause Probable} & \textbf{Solution Rapide} \\
\hline
\endhead

404 sur \texttt{http://127.0.0.1:5000/} & Aucune route \texttt{/} définie & Ajouter la route \texttt{@app.route("/")} + \texttt{render\_template("dashboard.html")} \\
\hline

404 sur \texttt{/nchain\_balance/...} & Route pas définie / faute de frappe & Vérifier la fonction \texttt{nchain\_balance} et l'URL exacte \\
\hline

Erreur Mirror Node dans JSON & Mauvais token ID, mauvais réseau, objet au lieu de string & Mettre \texttt{token\_id} en string, vérifier \texttt{MIRROR\_BASE} et \texttt{0.0.10136204} \\
\hline

Carte ``Trésorerie NCHAIN'' reste vide & Script non chargé ou ID différent & Vérifier présence de \texttt{id="nchain-balance"} et du script \texttt{loadNchainBalance()} \\
\hline

Page dashboard blanche & Mauvaise route ou template manquant & Vérifier \texttt{render\_template("dashboard.html")} et présence du fichier \\
\hline

\end{longtable}

% =====================================================
\newpage
\section{Checklist Technique de Démarrage}
% =====================================================

\subsection{1. Préparer l'environnement}

\begin{itemize}
  \item[$\square$] Activer le venv Python du projet, si utilisé (\texttt{source venv/bin/activate})
  \item[$\square$] Se placer dans le dossier du projet : \texttt{cd \textasciitilde/nutrichain\_restored}
  \item[$\square$] Vérifier que les dépendances sont installées : \texttt{pip install -r requirements.txt}
\end{itemize}

\subsection{2. Vérifier la config Hedera / Mirror Node}

\begin{itemize}
  \item[$\square$] Confirmer le réseau utilisé dans \texttt{api.py} : \texttt{HEDERA\_NETWORK = "testnet"} ou \texttt{"mainnet"}
  \item[$\square$] Vérifier la base URL Mirror Node selon le réseau (ex. \texttt{https://mainnet-public.mirrornode.hedera.com})
  \item[$\square$] Vérifier l'ID du token NCHAIN : \texttt{NCHAIN\_TOKEN\_ID = "0.0.10136204"} (en string pour les appels Mirror Node)
  \item[$\square$] Confirmer que \texttt{get\_nchain\_balance} appelle bien : \texttt{/api/v1/tokens/0.0.10136204/balances?account.id=0.0.10128148}
\end{itemize}

\subsection{3. Vérifier les routes Flask}

\begin{itemize}
  \item[$\square$] En haut de \texttt{api.py}, vérifier l'import : \texttt{from flask import Flask, jsonify, request, render\_template}
  \item[$\square$] Vérifier la route racine (dashboard) : \texttt{@app.route("/")} avec \texttt{render\_template("dashboard.html")}
  \item[$\square$] Vérifier la route API NCHAIN : \texttt{@app.route("/nchain\_balance/<account\_id>")}
  \item[$\square$] Vérifier le bloc de lancement : \texttt{if \_\_name\_\_ == "\_\_main\_\_"}: \texttt{app.run(host="0.0.0.0", port=5000, debug=True)}
\end{itemize}

\subsection{4. Lancer et tester backend}

\begin{itemize}
  \item[$\square$] Lancer l'API : \texttt{python api.py}
  \item[$\square$] Confirmer dans les logs : ``Running on http://127.0.0.1:5000''
  \item[$\square$] Tester l'API NCHAIN directement : \texttt{http://127.0.0.1:5000/nchain\_balance/0.0.10128148}
  \item[$\square$] Vérifier que la réponse JSON contient \texttt{balance}, \texttt{symbol}, \texttt{token\_id}
\end{itemize}

\subsection{5. Vérifier le dashboard HTML}

\begin{itemize}
  \item[$\square$] Confirmer que \texttt{templates/dashboard.html} contient la grille de stats avec la carte NCHAIN
  \item[$\square$] Confirmer la présence de \texttt{<div class="stat-card">} avec \texttt{id="nchain-balance"}
  \item[$\square$] En bas de \texttt{dashboard.html}, vérifier le script \texttt{loadNchainBalance()}
  \item[$\square$] Le script doit être juste avant \texttt{</body>}
\end{itemize}

\subsection{6. Tester le dashboard en navigateur}

\begin{itemize}
  \item[$\square$] Ouvrir le dashboard : \texttt{http://127.0.0.1:5000/}
  \item[$\square$] Vérifier visuellement la carte ``Trésorerie NCHAIN''
  \item[$\square$] Si vide : ouvrir la console (F12 $\Rightarrow$ Console) et regarder les erreurs
  \item[$\square$] Vérifier qu'il n'y a pas d'erreur ``Error loading NCHAIN balance''
\end{itemize}

\subsection{7. En cas de bug}

\begin{itemize}
  \item[$\square$] Si 404 sur \texttt{/} $\Rightarrow$ vérifier la route \texttt{@app.route("/")}
  \item[$\square$] Si 404 sur \texttt{/nchain\_balance/...} $\Rightarrow$ vérifier l'URL exacte et la définition de la route
  \item[$\square$] Si erreur Mirror Node $\Rightarrow$ vérifier \texttt{MIRROR\_BASE}, \texttt{NCHAIN\_TOKEN\_ID} et le réseau
  \item[$\square$] Si le dashboard ne voit pas les nouvelles modifications HTML $\Rightarrow$ vider le cache navigateur (Ctrl+F5)
  \item[$\square$] Vérifier qu'on édite bien \texttt{templates/dashboard.html} (pas une copie)
\end{itemize}

% =====================================================
\section{Lancement du Projet}
% =====================================================

\subsection{Commandes essentielles}

\begin{lstlisting}[language=bash]
# 1. Se placer dans le projet
cd ~/nutrichain_restored

# 2. Activer le venv (si utilisé)
source venv/bin/activate

# 3. Installer les dépendances (première fois)
pip install -r requirements.txt

# 4. Lancer l'API
python api.py

# 5. Ouvrir le navigateur
# http://127.0.0.1:5000/              Dashboard
# http://127.0.0.1:5000/nchain_balance/0.0.10128148    API (test)
\end{lstlisting}

\subsection{URLs utiles}

\begin{itemize}
  \item Dashboard : \texttt{http://127.0.0.1:5000/}
  \item API NCHAIN (test rapide) : \texttt{http://127.0.0.1:5000/nchain\_balance/0.0.10128148}
\end{itemize}

Mirror Node est uniquement utilisé en lecture seule pour les tokens NCHAIN (solde).

% =====================================================
\section{Notes}
% =====================================================

\begin{itemize}
  \item Hedera est actuellement en mode \textbf{lecture seule}. Pas de clé privée sur le serveur.
  \item Le compte trésorier \texttt{0.0.10128148} détient 100 000 000 NCHAIN.
  \item Le token NCHAIN ID est \texttt{0.0.10136204} sur mainnet.
  \item Consulter \url{https://docs.hedera.com} pour l'API Mirror Node complète.
\end{itemize}

\end{document}